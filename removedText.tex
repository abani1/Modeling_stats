\textbf{Not sure if we have to include here all of them - here we should present DATA}

A third case study is a rain-triggered debris avalanche down the channels of Nevado de Colima (Mexico).
A debris avalanche is the failure of a major part of an established volcanic edifice. These usually involve predominantly cold and mechanically weakened rocks as well as large volumes, up to tens of cubic kilometers. The triggering mechanism can be internal (such as the intrusion of new magma and pressurization) or external (such as seismic acceleration, rainfall, etc.). As such, the material involved in these events can be a mixture or fresh magmatic components as well as ancient hydrothermally weakened rocks. The material volume is usually in the range of $10^4$ -- 3$\times10^{10}$ [$\mathrm{m^3}$]. These types of flows are distinguished in that they involve large parts of the edifice that are not thoroughly broken up, i.e. some particles can be up to hundreds of meters or even kilometers in diameter.

A fourth case study is the pumice flow generated by a column collapse in the Mono-Inyo chain area (CA).
A column collapse pyroclastic flow is formed from a vertically-ejected eruption column. Instead of rising in the eruption plume, denser than air material collapses back to the ground and begins to flow as a gravity current. The kinetic energy of the material that lands on the upper slopes of the volcano is therefore redirected into downslope motion. These types of flow involve ash and low density pumiceous material, formed by vesiculation and fragmentation of gas-rich magmas in the volcanic conduit during the explosions. In a column collapse, during the horizontal component of flow, ambient air is entrained into the flow and heated. The flows therefore involve a mixture of hot gas, ash, and small pumice clasts (a few [cm] to 0.5 [m]). This extremely hot mixture can have initial downslope speeds in the range of 30 -- 80 meters per second. The volume of material involved in such a flow is roughly constrained to the volume of the plug and upper conduit (for short-lived explosive eruptions) and therefore will most likely be in the range $10^4$ -- $10^6$ [$\mathrm{m^3}$]. 