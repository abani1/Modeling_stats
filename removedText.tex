\textbf{Not sure if we have to include here all of them - here we should present DATA}

A third case study is a rain-triggered debris avalanche down the channels of Nevado de Colima (Mexico).
A debris avalanche is the failure of a major part of an established volcanic edifice. These usually involve predominantly cold and mechanically weakened rocks as well as large volumes, up to tens of cubic kilometers. The triggering mechanism can be internal (such as the intrusion of new magma and pressurization) or external (such as seismic acceleration, rainfall, etc.). As such, the material involved in these events can be a mixture or fresh magmatic components as well as ancient hydrothermally weakened rocks. The material volume is usually in the range of $10^4$ -- 3$\times10^{10}$ [$\mathrm{m^3}$]. These types of flows are distinguished in that they involve large parts of the edifice that are not thoroughly broken up, i.e. some particles can be up to hundreds of meters or even kilometers in diameter.

A fourth case study is the pumice flow generated by a column collapse in the Mono-Inyo chain area (CA).
A column collapse pyroclastic flow is formed from a vertically-ejected eruption column. Instead of rising in the eruption plume, denser than air material collapses back to the ground and begins to flow as a gravity current. The kinetic energy of the material that lands on the upper slopes of the volcano is therefore redirected into downslope motion. These types of flow involve ash and low density pumiceous material, formed by vesiculation and fragmentation of gas-rich magmas in the volcanic conduit during the explosions. In a column collapse, during the horizontal component of flow, ambient air is entrained into the flow and heated. The flows therefore involve a mixture of hot gas, ash, and small pumice clasts (a few [cm] to 0.5 [m]). This extremely hot mixture can have initial downslope speeds in the range of 30 -- 80 meters per second. The volume of material involved in such a flow is roughly constrained to the volume of the plug and upper conduit (for short-lived explosive eruptions) and therefore will most likely be in the range $10^4$ -- $10^6$ [$\mathrm{m^3}$]. 












A traction-free boundary condition is always imposed at the free surface $F^s=0$, while a sliding law at the interface between the granular flow and the basal surface is imposed:
\begin{equation}\label{eq:SurTracBC}
\underline{\underline{{\mathbf{T}}}}^s\underline{n}^s=\underline{0}, \ \ at \ F^s(x,y,t)=0
\end{equation}
\begin{equation}\label{eq:BedTracBC}
\underline{\underline{{\mathbf{T}}}}^b\underline{n}^b-\underline{n}^b(\underline{n}^b\cdot \underline{\underline{{\mathbf{T}}}}^b\underline{n}^b)=\frac{\underline{{\textbf u}}^b}{\Vert\underline{{\textbf u}}^b\Vert}(\underline{n}^b\cdot \underline{\underline{{\mathbf{T}}}}^b\underline{n}^b)\mu_{\mathrm{Bed}}, \ \ at \ F^b(x,y,t)=0
\end{equation}
where $\mu_{\mathrm{Bed}}$ is called the basal friction coefficient and  $\underline{n}^b=\nabla F^b/\Vert\nabla F^b\Vert$ is the normal unit vector to the basal surface.


It is worth mentioning that $\underline{\underline{{\mathbf{T}}}} \ \underline{n}$ is a resisting traction, while $\underline{n}\cdot\underline{\underline{{\mathbf{T}}}} \ \underline{n}$ is the normal pressure and $\underline{n}\cdot\underline{\underline{{\mathbf{T}}}}-\underline{n}(\underline{n}\cdot\underline{\underline{{\mathbf{T}}}} \ \underline{n})$ is the resulting shear traction. Considering Equation (\ref{eq:BedTracBC}), Coulomb dry-friction sliding law expresses that the magnitude of the basal shear stress is equal to the normal basal pressure multiplied by the basal friction coefficient, $\mu_{\mathrm{Bed}}$. Furthermore, Equation (\ref{eq:BedTracBC}) is stating that the shear traction is assumed to point in the opposite direction to the basal velocity, $\underline{{\textbf u}}^b$ (resisting shear traction) which also implicitly assumes that $\underline{{\textbf u}}^b\cdot \underline{n}^b=0$ when we also consider the basal kinematic boundary condition.














The TITAN2D toolkit is based on a depth-averaged model for an incompressible continuum, or a \emph{shallow-water} approximation. The basic incompressible conservation equations for mass and momentum describing the motion of an avalanching mass are:
\begin{eqnarray}
\label{eq:N_S}
\nabla \cdot \underset{^\sim}{\textbf u} &=& 0 \nonumber \\
\frac{\partial}{\partial t}(\rho \ \underset{^\sim}{\textbf u})+\nabla \cdot (\rho \ \underset{^\sim}{\textbf u}\otimes\underset{^\sim}{\textbf u}) &=& \nabla \cdot \underset{^\sim}{\sigma}+\rho \ \underset{^\sim}{\textbf g}
\end{eqnarray}
Where $\underset{^\sim}{\textbf u}$, is the material velocity vector, $\rho$ is its constant density, $\underset{^\sim}{\sigma}$ is the Cauchy stress tensor and  $\underset{^\sim}{\textbf g}$ is the gravity vector. Based on the material rheology we choose, the Cauchy stress tensor is defined differently \cite{FreundtBursik1998}, and the resulting system of hyperbolic PDEs are solved with the aforementioned solver \cite{Patra2005}.

%
%were first modified to accommodate granular mass flows by Eglit and Sveshnikova \cite{eglit1980mms}. They used the equations to study snow avalanches.
%In 1989, Savage and Hutter\cite{SavageHutter} became the first to model geophysical mass flows with depth-averaged equations. Their original model has been improved upon by them and others since then \cite{Hutter1993,Iverson1997,Gray1999,Iverson2001,PudasainiHutter2003,SavageIverson2003}. The Saint-Venant equations are a “shallow-water” model derived from the continuum flow equations by depth-averaging. They are commonly used to model various kinds of flooding, such as storm surge, wave run up, and overland flows. They can even model the world’s oceans (which can be considered shallow when compared to the radius of the earth) as a single body of water. The “shallow-water” equations






Granular avalanches exhibit a shallow flow geometry, requiring that the shallowness parameter, $\epsilon\triangleq h/l$, defined by the ratio of characteristic flow height, $h$, ($\mathcal{O}(1) \ m$) and the characteristic length scale, $l$, ($\mathcal{O}(10^3) \ m $) is small, $\epsilon \ll 1$. This geometrical property justifies a model formulation in terms of depth-averaged field variables \cite{SavageHutter,Bartelt1999}. Details on the integration of continuum models for shallow free surface flows can be studied in \cite{SavageHutter,PudasainiHutter2007}. The ``depth-averaged Saint-Venant'' equations were first modified to accommodate granular mass
flows by Eglit and Sveshnikova to study snow avalanches \cite{eglit1980mms}. In 1989, Savage and Hutter \cite{SavageHutter} became the first to model geophysical mass flows with depth-averaged equations. Their original model has been improved upon by them and others since then \cite{Hutter1993,Iverson1997,Gray1999,Iverson2001,PudasainiHutter2003,SavageIverson2003}.


\begin{figure}[H]
        \centering
        \includegraphics[width=.5\textwidth]{Figures/coord.jpg}
        \caption{The local Cartesian coordinate system, the \textit{z} direction is normal to the
surface; the \textit{x} and \textit{y} directions are tangent to the surface and roughly aligned
with East and North respectively.}
        \label{coordinate}
\end{figure}




\subsection{Initial conditions}\label{IC}

TITAN2D can be initialized with multiple sources of granular material moving at an initial velocity $\|\underset{^\sim}{\bar{\textbf{u}}_0}\| \geq 0$ along a specified direction, which can be defined as either a fixed-volume initial \emph{pile} or a \emph{flux-source} (a specified region where material is already flowing). The initial granular mass is driven by gravitational and hydrostatic pressure gradient forces, while its motion is resisted by dissipation within the flowing material itself or with the bed.   Turbulent dissipation  could also occur.  It is generally thought that all the dissipation mechanisms can be accounted for by a judicious choice of material rheology. In the current release of TITAN2D, we have therefore added support for different rheological models.



The mathematical modeling used in TITAN2D requires that the chosen value of the internal friction angle be greater than the bed friction angle. Fortunately, apart from this requirement, the output of TITAN2D tends to be fairly insensitive to reasonable variation of the internal friction angle, with larger values providing increased resistance to spreading of the material.







In 1998, \cite{DadeHuppert1998} noted that field data for some  natural flows suggest that a simple, constant retarding stress model could represent the rheology of the flowing material. Their proposed rheology model represents the simplest type of  plastic material behavior. Inspired by Dade and Huppert, \cite{Kelfoun2005} and \cite{Kelfoun2009} applied a constant retarding stress model for pyroclastic flow simulations in two specific eruptions. These works suggested that the plastic model gives more realistic results in matching flow inundation areas and velocites than do friction-based rheology models, such as Mohr-Coulomb and Pouliquen-Forterre models.

Having been motivated by numerical studies \cite{Iordanoff2004,DaCruz2005} and experimental work \cite{PouliquenForterre2002,Forterre2003}, \cite{Jop2006} proposed a new constitutive relation for dense granular flows. It is assumed that flowing granular matter shares similarities with classical visco-plastic fluids such as Bingham fluids. Based on a new set of experiments, they suggested that this simple visco-plastic rheology can capture granular flow
properties, and serve as a basic tool in modeling more complex flows in geophysical or industrial applications. Recently, \cite{Gray2014,Edwards2015} studied Jop's visco-plastic model in the framework of depth-averaged conservation of mass and momentum.











The empirical dependence between the ratio of the flow thickness $h$ to $h_{stop}(\phi)$  and 


, is:
\begin{equation}\label{h_stop}
h_{stop}(\phi)=\beta \frac{h}{Fr}
\end{equation}




The main advantage of the depth-averaged Saint-Venant equations is that the dynamics of the flowing layer can be predicted without the detailed knowledge of the internal structure of the flow. The complex 3D rheology of the material is mainly embedded in a basal friction term $\mu_b$. Taking a simple constant Coulomb-like basal friction is sometimes sufficient to capture the main flow characteristics and has been used to describe granular slumping


,Mangeney2005}, rapid flows down smooth inclines , and shock waves . However, for flows down rough slopes, the assumption of a constant solid friction is not compatible with the observation of steady uniform flows over a range of inclination angles. The more complex basal friction laws, $\mu_{b}(\|\underset{^\sim}{\bar{\textbf{u}}}\|,h)$, we introduced would be more sofisticated choice for quantitative predictions in complex situations such as a propagating steady front, mass spreading, or surface instabilities \cite{Forterre2003,,Pouliquen1999,PouliquenForterre2002}.









 